\documentclass[12pt,a4paper]{article}
\usepackage{geometry}
\usepackage{titlesec}
\usepackage{graphicx}
\usepackage{enumitem}
\usepackage{setspace}
\usepackage{hyperref}

% 页面设置
\geometry{a4paper, left=3cm, right=2.5cm, top=3cm, bottom=2.5cm}
\linespread{1.5}  % 行距

% 标题格式
\titleformat{\section}{\Large\bfseries}{\thesection}{1em}{}
\titleformat{\subsection}{\large\bfseries}{\thesubsection}{1em}{}

\begin{document}

% 标题页
\begin{center}
    \vspace*{2cm}
    \textbf{\LARGE 毕业论文中期检查报告}\\[1.5cm]
    \textbf{论文题目:} 分形函数水平集的基本性质\\[0.5cm]
    \textbf{学生姓名:} 楼康杰\\[0.5cm]
    \textbf{学号:} 3210103408\\[0.5cm]
    \textbf{指导教师:} 阮火军教授\\[0.5cm]
    \textbf{学院:} 数学科学学院\\[0.5cm]
    \textbf{日期:} \today
\end{center}
\newpage

% 研究进展
\section{研究进展}
本论文研究分形函数水平集的基本性质,目前已完成以下内容:
\begin{enumerate}
    \item 阅读并整理相关文献,包括 Han Y. (2020), Allaart \\& Kawamura (2011) 等,掌握当前研究进展。
    \item 初步分析 Takagi 函数的水平集性质,并尝试给出相关的数学描述。
    \item 编写部分理论证明,验证某些基本性质的正确性。
    \item 进行数值模拟,辅助理论分析。
\end{enumerate}

% 主要问题
\section{目前存在的问题}
在研究过程中,遇到以下问题:
\begin{itemize}
    \item 某些定理的证明较复杂,需要进一步细化推理步骤。
    \item 数值模拟部分的计算效率较低,需优化算法。
    \item 参考文献较多,需进一步整理归纳。
\end{itemize}

% 后续计划
\section{后续研究计划}
未来的研究计划如下:
\begin{enumerate}
    \item 继续完善理论分析,尝试改进证明方法。
    \item 深入分析水平集的几何结构,探讨其维度特性。
    \item 优化数值模拟,提高计算效率,并对实验结果进行详细分析。
    \item 结合前人研究,整理已有结论,并撰写论文初稿。
\end{enumerate}

% 参考文献
\section{参考文献}
\begin{enumerate}
    \item Y. Han, Weak tangent and level sets of Takagi functions, Monatsh. Math., 192 (2020), 249--264.
    \item P. Allaart, K. Kawamura: The Takagi function: a survey. Real Anal. Exch. 37 (2011), 1--54.
    \item J.M. Fraser, Assouad dimension and fractal geometry, Cambridge University Press, 2021.
    \item K. Falconer, Fractal geometry: Mathematical foundations and applications, 3rd ed., John Wiley \& Sons, 2014.
\end{enumerate}

\end{document}

